\documentclass[10pt, oneside]{article}   	
\usepackage[margin=0.8in]{geometry}  
\usepackage[dvips]{graphics}
\usepackage{epsfig}
\usepackage{amsmath}
\usepackage{xspace}
\usepackage{fancybox}
\usepackage{graphicx,xspace,cite,verbatim,comment}
\usepackage[demo]{graphicx}
\usepackage{caption}
\usepackage{subcaption}
\usepackage{hyperref,array,color,balance,multirow}
\usepackage{balance,float,url,amsfonts,alltt}
\usepackage{mathtools,rotating,amsmath,amssymb}
\usepackage{color,cite,ifpdf,fancyvrb,array,listings}
\usepackage{algorithm,algpseudocode}
\usepackage{tabularx}
\usepackage{natbib}              		
\geometry{letterpaper}                   		
\usepackage{graphicx}		
\usepackage{subcaption}		
\usepackage{amssymb}

%SetFonts

%SetFonts


\title{\textbf{CS 638 Project Stage 2 Report}}
\author{Zhiwei Fan\hspace{7ex}
	   Lingfeng Huang\hspace{7ex}
	   Fang Wang\\
	   zfan29@wisc.edu\hspace{3ex}
	   lhuang58@wisc.edu\hspace{3ex}
	   fwang64@wisc.edu
	   }
%\date{}							% Activate to display a given date or no date

\begin{document}
\maketitle 

\section*{Source Websites and Relevant Documentation}
\begin{enumerate}
  	\item Source Websites:\\
		\textbf{Barnes\&Noble}: \textit{http://www.barnesandnoble.com}\\
		\textbf{Goodreads}: \textit{https://www.goodreads.com}
		
	\item Corresponding Tables:\\
		\textbf{tableA.csv}: table transformed from \textbf{Barnes\&Noble}'s data\\
		\textbf{tableB.csv}: table transformed from \textbf{Goodreads}'s data
		
	\item Original Schema:\\
		\textbf{tableA}(\underline{id}: integer, title: string, year: integer, pages: integer, day: integer, publisher: string, month: integer, authors: string, isbn3: string)\\
		\textbf{tableB}(\underline{id}: integer, title: string, isbn: string, pageCount: integer, author: string, publisher: string, date: string)	
\end{enumerate}

\section*{Schema Transformation}
According to \textit{Original Schema} in previous section, tableA and tableB originally have different schemas. After careful consideration, we decide to transform both tables to 
be in the consistent schema shown as the following: 
\vspace{1ex}
 \\
 \textbf{table}(\underline{id}: integer, title: string, author: string, publisher: string, pages: integer, year: integer, month: integer, day: integer, isbn: string)
 
 \subsection*{Specific Steps of Transformation}
 Most attributes in \textit{tableA} remain the same. The only change made to \textit{tableA} was \textit{renaming} attribute \texit{isbn3} to \textit{isbn}. For \textit{B}, the attribute
 \textit{pageCount} was renamed to \textit{page}; attribute \textit{date} has been properly parsed and split into three separated attributes \textit{year}, \textit{month} and \textit{day}
 by running appropriate python scripts on the original csv files \textit{tableA.csv} and \textit{tableB.csv}.
 
 \subsection*{Set of Attribute: S}
 According to the transformation steps we have described in the previous part, the attributes of the consistent schema in \textit{set S} is: \\
\textit{ id, title, author, publisher, pages, year, month, day, isbn}.
 



\end{document}  